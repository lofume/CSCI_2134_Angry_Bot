\documentclass[12pt,pdftex]{article}

\usepackage{fullpage}
\usepackage{graphicx}
\usepackage{fancyhdr}
\usepackage{url}

\begin{document}
\newcommand{\Term}{Fall 2015}
\newcommand{\Course}{CSCI1110}
\newcommand{\Assignment}{TimBot Specification}
\newcommand{\Due}{12 noon, Monday, October 19, 2015}

\renewcommand{\headheight}{15pt}
\renewcommand{\headsep}{5pt}
\pagestyle{fancy}
\lhead{\bf \Course}
\chead{\bf \Term}
\rhead{\bf \Assignment}
\lfoot{}
\cfoot{\thepage}
\rfoot{}

\title{%\vspace*{-12ex}
       \Assignment}
%\author{Instructor: Alex Brodsky}
\date{}
\maketitle

\newcommand{\ignore}[1]{}
\newcommand{\docommand}[1]{\begin{center} \fbox{\bf\tt #1} \end{center}}
\newcommand{\NoItemSpace}{\setlength{\parskip}{0pt}
                          \setlength{\partopsep}{0pt}
                          \setlength{\parsep}{0pt}
                          \setlength{\itemsep}{0pt}}

%\vspace*{-8ex}
This is a robot simulation where robots compete for supremacy of
the Doh-Nat Planet.

\section*{Background}
TimBots live on Doh-Nat.  A torus-shaped world that has limited
resources.  TimBots need power to survive, but the only power source
on Doh-Nat is Spresso, a plant that is neither plentiful, nor
fast-growing.  Hence, the TimBots have concluded that eliminating
each other is the best way to ensure their own survival.

The surface of Doh-nat is an $m \times n$ grid, where the right
edge of the grid is adjacent to the left edge, and the top edge is
adjacent to the bottom edge.  Each square in the grid is called a
district, and grows a limited number of Spresso plants.  One
district-worth of Spresso plants provides a TimBot with a few Jolts
of energy, which a TimBot needs to function.  Once the Spresso has
been harvested from a district, it takes several days to grow back.
To survive, the TimBots must roam the planet, feeding on Spresso,
and eliminating their competitors in the process.

A TimBot can move from one district to another, harvest Spresso,
fire its ion cannon at its competitors, and defend itself with a
force-shield.  Each of actions costs the TimBot one Jolt of energy.
If a TimBot runs out of energy, it becomes nonfunctional.

\section*{Rules of the Simulation} 
The simulation takes place on an $m \times n$ torus grid, where the
left edge is adjacent to the right edge, and the top edge is adjacent
to bottom edge.  I.e., it is impossible to fall off the grid---moving
across one edge of the grid moves the TimBot to the other side of
the grid.

The simulation takes place over a number of rounds.  Each round consists,
of seven phases:
\begin{enumerate}\NoItemSpace
\item {\bf Start} each round by using one Jolt of energy to power the TimBot
      through the rest of the round. (Additional energy may be required to
      perform various actions.  If a TimBot does not have a Jolt of energy,
      it becomes nonfunctional.

\item {\bf Sense} the districts to the north, south, east, and west of
      their present district.  The sensors can sense (i) the ripeness
      of Spresso plants in a district, and (ii) if there is a TimBot
      in a district.

\item {\bf Fire} its ion cannon zero or more times (as long as the
      TimBot has the energy to do so).  One ion shot requires one
      Jolt of energy.  The ion cannon has a range of one district,
      to the north, south, east, or west of the TimBot.  

\item {\bf Defense} from ion cannon fire.  If a TimBot detects an 
      ion attack, it can defend itself with its force shield (as
      long as it has the energy to do so).  It requires one Jolt
      of energy to stop one ion shot.  If a TimBot does not have
      the energy to stop an ion shot, it is rendered nonfunctional.

\item {\bf Move} to another district.  A TimBot may move to the
      district north, south, east, or west of their present location.
      A move takes one Jolt of energy.   A TimBot may also choose not
      to move.

\item {\bf Battle} within a district.  If two or more TimBots are
      in the same district, they use their shields to try to push
      the other TimBots out.  The TimBots keep raising their shields
      until they run out of energy and become nonfunctional, or the
      other TimBot(s) run out of energy first.  Hence, the
      TimBot with the greatest amount of energy survives and all
      the other TimBots become nonfunctional.  The winner's energy
      level is decreased by the energy level of the runner up.
      Hence, if there is a tie for highest energy level, all
      combatants become nonfunctional.

\item {\bf Harvest} the district (if there are Spresso plants to harvest).
      Once the Spresso plants are harvested, it takes a number of rounds
      to grow the Spresso plants.
\end{enumerate}
If a TimBot's level drops below 0, they become nonfunctional.


\section*{Types of TimBots}
There are four different personalities of TimBots:
\begin{description}\NoItemSpace
\item[ChickenBot]: (Personality C)  
     If a ChickenBot senses a TimBot in an adjoining
     district, it will move to an empty district that (preferably)
     has Spresso to harvest.  That is, during the Move Phase, if
     the Chicken Bot has one or more Jolts of energy, it will
     consider all five districts (CURRENT, NORTH, EAST, SOUTH, and
     WEST), and will order them according to the following criteria 
     (in order of preference):
     \begin{enumerate}\NoItemSpace
     \item There are no other TimBots in the district.
     \item There are no TimBots in the adjacent districts.
     \item The least number of rounds before Spresso plants are ready for
           harvest.
     \item The order the districts as listed above.
     \end{enumerate}
     It will then move to the most preferred district.  The ChickenBot
     will never fire its ion cannon, preferring to conserve energy.

\item[SpressoBot]: The SpressoBot will always follow the Spresso plants.
     It will try to move to a district where it can harvest the Spresso 
     plants as quickly as possible.  That is, during the Move Phase, if
     the SpressoBot has one or more Jolts of energy, it will consider all 
     five districts (CURRENT, NORTH, EAST, SOUTH, and WEST), and will 
     order them according to the following criteria (in order of preference):
     \begin{enumerate}\NoItemSpace
     \item The least number of rounds before Spresso plants are ready for
           harvest.
     \item There are no other TimBots in the district.
     \item There are no TimBots in the adjacent districts.
     \item The order the districts as listed above.
     \end{enumerate}
     It will then move to the most preferred district.  The SpressoBot
     will never fire its ion cannon, preferring to conserve energy.

\item[AngryBot]: The AngryBot will attack other TimBots.
     It will try to move to a district where it can attack another TimBot
     and, where there are Spresso plants ready for harvest (preferably).
     That is, during the Move Phase, if the AngryBot has three or more Jolts 
     of energy, it will consider all five districts (CURRENT, NORTH, EAST, 
     SOUTH, and WEST), and will order them according to the following 
     criteria (in order of preference):
     \begin{enumerate}\NoItemSpace
     \item There are other TimBots in the district.
     \item The least number of rounds before Spresso plants are ready for
           harvest.
     \item The order the districts are listed above.
     \end{enumerate}
     It will then move to the most preferred district.  

     If the AngryBot had less than three Jolts of energy, it behaves
     like a SpressoBot.  An AngryBot is a subclass of
     SpressoBot.) 

\item[BullyBot]: A BullyBot is a gun-happy ChickenBot.  (A 
     BullyBot is a subclass of ChickenBot.)  During the Fire Phase,
     the BullyBot will shoot its cannon once into every adjoining district 
     that has a TimBot in it, in the order NORTH, EAST, SOUTH,
     WEST, as long as long as it has three or more Jolts of energy.  I.e.,
     if its energy level drops to 2, it stops shooting. Otherwise, it 
     behaves like a ChickenBot.
\end{description}


\section*{The Simulation}
The simulator reads in the simulation configuration and then
instantiates and runs the the simulation, printing out the state
of the simulation after each round.  The simulation continues until
the required number of rounds have transpired.

\subsection*{Input}
The input format to the simulator is as follows: The input comprises 
six integers, follow by one or more TimBot configurations.  The
first six integers are:
\begin{quote}
\begin{description}\NoItemSpace
\item[R]: number of rows in the grid representing planet DohNat
\item[C]: number of columns in the grid representing planet DohNat
\item[J]: number of jolts that Spresso harvest yields
\item[G]: number of rounds needed for Spresso to grow after harvest
\item[N]: maximum number of rounds to run the simulation for
\item[T]: number of TimBot configurations to follow.
\end{description}
\end{quote}

A TimBot configuration consists of one string and four integers.
\begin{quote}
\begin{description}\NoItemSpace
\item[P]: string denoting personality of TimBot.  E.g.,
          {\tt chicken}, {\tt spresso}, {\tt angry}, {\tt bully}.
\item[I]: the numeric ID of the TimBot.  (All IDs are assumed to be unique.)
\item[X]: the $x$ coordinate of the TimBot in the grid, where  $0 \leq X < C$
\item[Y]: the $y$ coordinate of the TimBot in the grid, where  $0 \leq Y < R$
\item[E]: number of Jolts the TimBot initially has at the start of 
          the simulation.
\end{description}
\end{quote}
All values are separated by white-space.  

\subsection*{Semantics}
Upon reading the configuration, the simulator instantiates a {\it
DohNat} using arguments $R$, $C$, $J$, and $G$.  The simulator
creates a list of {\it TimBot} objects, one for each configuration,
and adds the objects to the {\it DohNat} object.  The simulator
then loops for at most $N$ times, executing a round during each
iteration.  At the end of each round, the simulator outputs the
state of the similation and traverses the list of {\it TimBot}
objects to count all functional TimBots.  If the number of functional
TimBots is one or less, the simulator stops the loop.

\subsection*{Output}
After each round completes, the simulator prints out to the 
console the string {\tt Round $r$}, where $r$ is the current round,
followed by the grid of {\it DohNat}.  The format is:
\begin{quote}
\noindent {\tt Round $r$\\
\noindent $\begin{array}{lllll}
D_{0,0} & D_{1,0} & D_{2,0} & \ldots &  D_{C-1,0}\\
D_{0,1} & D_{1,1} & D_{2,1} & \ldots &  D_{C-1,1}\\
D_{0,2} & D_{1,2} & D_{2,2} & \ldots &  D_{C-1,2}\\
\vdots  & \vdots  & \vdots  & \ddots  & \vdots \\
D_{0,R-1} & D_{1,R-1} & D_{2,R-1} & \ldots &  D_{C-1,R-1}\\
\end{array}$}
\end{quote}
where $D_{x,y}$ is the district at coordinates $(x,y)$, and uses the format
specified by the {\it District} class.

\subsection*{Sample Input }
\begin{quote}
\begin{verbatim}
5 5 5 10 100 4
chicken 1 2 2 1
spresso 2 3 3 1
angry 3 2 3 1
bully 4 3 2 1
\end{verbatim}
\end{quote}

\subsection*{Sample Output}
\begin{quote}
\begin{verbatim}
Round 0
|           0||           0||           0||           0||           0|
|           0||           0||           0||           0||           0|
|           0||           0||(C  1  5) 10||(B  4  5) 10||           0|
|           0||           0||(A  3  5) 10||(S  2  5) 10||           0|
|           0||           0||           0||           0||           0|
Round 1
|           0||           0||           0||           0||           0|
|           0||           0||(C  1  7) 10||(B  4  6) 10||           0|
|           0||           0||(A  3  3)  9||           9||           0|
|           0||           0||           9||           9||(S  2  7) 10|
|           0||           0||           0||           0||           0|
Round 2
|           0||           0||(C  1  9) 10||(B  4  8) 10||           0|
|           0||           0||           9||           9||           0|
|           0||(A  3  6) 10||           8||           8||(S  2 10) 10|
|           0||           0||           8||           8||           9|
|           0||           0||           0||           0||           0|
Round 3
|           0||           0||           9||           9||           0|
|           0||(A  3  9) 10||           8||           8||(S  2 13) 10|
|           0||           9||           7||           7||           9|
|           0||           0||           7||           7||           8|
|           0||           0||(C  1 11) 10||(B  4 10) 10||           0|
Round 4
|           0||(A  3 12) 10||           8||           8||(S  2 16) 10|
|           0||           9||           7||           7||           9|
|           0||           8||           6||           6||           8|
|           0||           0||           6||           6||           7|
|           0||(C  1 13) 10||           9||           9||(B  4 12) 10|
Round 5
|(S  2 18) 10||           9||           7||           7||           9|
|           0||           8||           6||           6||           8|
|           0||           7||           5||           5||           7|
|           0||(C  1 16) 10||           5||           5||           6|
|(B  4 14) 10||(A  3 10)  9||           8||           8||           9|
Round 6
|           9||           8||           6||           6||           8|
|(S  2 20) 10||           7||           5||           5||           7|
|           0||           6||           4||           4||           6|
|(C  1  8) 10||(A  3  7)  9||           4||           4||           5|
|           9||           8||           7||           7||           8|
Round 7
|           8||           7||           5||           5||           7|
|           9||           6||           4||           4||           6|
|(S  2 16) 10||           5||           3||           3||           5|
|(A  3  5)  9||           8||           3||           3||           4|
|           8||           7||           6||           6||           7|
Round 8
|           7||           6||           4||           4||           6|
|           8||           5||           3||           3||           5|
|(A  3  3)  9||(S  2 14)  4||           2||           2||           4|
|           8||           7||           2||           2||           3|
|           7||           6||           5||           5||           6|
Round 9
|           6||           5||           3||           3||           5|
|           7||           4||           2||           2||           4|
|           8||           3||(S  2 12)  1||           1||(A  3  1)  3|
|           7||           6||           1||           1||           2|
|           6||           5||           4||           4||           5|
Round 10
|           5||           4||           2||           2||           4|
|           6||           3||           1||           1||           3|
|           7||           2||(S  2 11)  0||           0||(A  3  0)  2|
|           6||           5||           0||           0||           1|
|           5||           4||           3||           3||           4|
Round 11
|           4||           3||           1||           1||           3|
|           5||           2||           0||           0||           2|
|           6||           1||(S  2 15) 10||           0||           1|
|           5||           4||           0||           0||           0|
|           4||           3||           2||           2||           3|
\end{verbatim}

\end{quote}

\end{document}
